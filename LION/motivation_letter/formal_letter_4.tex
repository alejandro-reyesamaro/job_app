%%%%%%%%%%%%%%%%%%%%%%%%%%%%%%%%%%%%%%%%%
% Professional Formal Letter
% LaTeX Template
% Version 1.0 (28/12/13)
%
% This template has been downloaded from:
% http://www.LaTeXTemplates.com
%
% Original author:
% Brian Moses (http://www.ms.uky.edu/~math/Resources/Templates/LaTeX/)
% with extensive modifications by Vel (vel@latextemplates.com)
%
% License:
% CC BY-NC-SA 3.0 (http://creativecommons.org/licenses/by-nc-sa/3.0/)
%
%%%%%%%%%%%%%%%%%%%%%%%%%%%%%%%%%%%%%%%%%

%----------------------------------------------------------------------------------------
%	PACKAGES AND OTHER DOCUMENT CONFIGURATIONS
%----------------------------------------------------------------------------------------

\documentclass[11pt,a4paper]{letter} % Specify the font size (10pt, 11pt and 12pt) and paper size (letterpaper, a4paper, etc)

\usepackage{graphicx} % Required for including pictures
\usepackage{microtype} % Improves typography
\usepackage{gfsdidot} % Use the GFS Didot font: http://www.tug.dk/FontCatalogue/gfsdidot/
\usepackage[T1]{fontenc} % Required for accented characters
\usepackage{wasysym}
\usepackage{marvosym}
\usepackage{ifsym}
%\usepackage[misc,geometry]{ifsym}
\usepackage[utf8]{inputenc}
\usepackage{lmodern} % load a font with all the characters


% Create a new command for the horizontal rule in the document which allows thickness specification
\makeatletter
\def\vhrulefill#1{\leavevmode\leaders\hrule\@height#1\hfill \kern\z@}
\makeatother

%----------------------------------------------------------------------------------------
%	DOCUMENT MARGINS
%----------------------------------------------------------------------------------------

\textwidth 6.75in
\textheight 9.5in
\oddsidemargin -.25in
\evensidemargin -.25in
\topmargin -1.5in
\longindentation 0.50\textwidth
\parindent 0.3in

%----------------------------------------------------------------------------------------
%	SENDER INFORMATION
%----------------------------------------------------------------------------------------

\def\Who{Alejandro REYES-AMARO} % Your name
\def\What{, PhD} % Your title
\def\Where{{\bf Laboratoire de Sciences du \\ Numeriques de Nantes}} % Your department/institution
\def\Address{\textifsymbol{18} 2 rue de la Houssini\`ere 44322 Nantes, France} % Your address
\def\CityZip{\hbox{} \hspace{4pt} 44322 Nantes, France} % Your city, zip code, country, etc
\def\Email{\Letter \hspace{0.1cm} alejandro.reyes@univ-nantes.fr} % Your email address
\def\TEL{\phone \hspace{0.1cm} (+33) 251-125831} % Your phone number
\def\MOBILE{\Mobilefone \hspace{0.1cm} (+33) 666-372524}
\def\URL{} % Your URL

%----------------------------------------------------------------------------------------
%	HEADER AND FROM ADDRESS STRUCTURE
%----------------------------------------------------------------------------------------

\address{
\includegraphics[width=1in]{logo.jpg} % Include the logo of your institution
\hspace{5.1in} % Position of the institution logo, increase to move left, decrease to move right
\vskip -1.07in~\\ % Position of the text in relation to the institution logo, increase to move down, decrease to move up
\Large\hspace{1.5in}{\sc Universit\'{e}} \hfill ~\\[0.05in] % First line of institution name, adjust hspace if your logo is wide
\hspace{1.5in}{\sc de Nantes} \hfill \normalsize % Second line of institution name, adjust hspace if your logo is wide
\makebox[0ex][r]{\bf \Who \What }\hspace{0.08in} % Print your name and title with a little whitespace to the right
~\\[-0.11in] % Reduce the whitespace above the horizontal rule
\hspace{1.5in}\vhrulefill{1pt} \\ % Horizontal rule, adjust hspace if your logo is wide and \vhrulefill for the thickness of the rule
\hspace{\fill}\parbox[t]{4in}{ % Create a box for your details underneath the horizontal rule on the right
\footnotesize % Use a smaller font size for the details
%\Who \\ \em % Your name, all text after this will be italicized
\em
%\Where\\ % Your department
\Address\\ % Your address
%\CityZip\\ % Your city and zip code
\TEL\\ % Your phone number
\MOBILE\\
\Email\\ % Your email address
%\URL % Your URL
}
\hspace{-1.4in} % Horizontal position of this block, increase to move left, decrease to move right
\vspace{-1in} % Move the letter content up for a more compact look
}

%----------------------------------------------------------------------------------------
%	TO ADDRESS STRUCTURE
%----------------------------------------------------------------------------------------
%\longindentation
\def\opening#1{\thispagestyle{empty}
{\centering\fromaddress \vspace{0.6in} \\ % Print the header and from address here, add whitespace to move date down
\hspace{6cm}13 Juin 2016\hspace*{\fill}\par} % Print today's date, remove \today to not display it
{\raggedright \toname \\ \toaddress \par} % Print the to name and address
\vspace{0.4in} % White space after the to address
\noindent #1 % Print the opening line
% Uncomment the 4 lines below to print a footnote with custom text
%\def\thefootnote{}
%\def\footnoterule{\hrule}
%\footnotetext{\hspace*{\fill}{\footnotesize\em Footnote text}}
%\def\thefootnote{\arabic{footnote}}
}

%----------------------------------------------------------------------------------------
%	SIGNATURE STRUCTURE
%----------------------------------------------------------------------------------------

\signature{{\bf \Who} } %\What} % The signature is a combination of your name and title

\long\def\closing#1{
\vspace{0.1in} % Some whitespace after the letter content and before the signature
\noindent % Stop paragraph indentation
\hspace*{\longindentation} % Move the signature right
\parbox{\indentedwidth}{\raggedright
#1 % Print the signature text
%\vskip 0.1in % Whitespace between the signature text and your name
\includegraphics[height=4.5\baselineskip]{pictures/sign}\\
\fromsig}} % Print your name and title

%----------------------------------------------------------------------------------------

\begin{document}

%----------------------------------------------------------------------------------------
%	TO ADDRESS
%----------------------------------------------------------------------------------------

\begin{letter}
{{\bf HD Technology}\\France\\
\vspace{0.2in}
\textbf{Object :} Candidature au poste "Docteur en Informatique" (HD Technology)
}

%----------------------------------------------------------------------------------------
%	LETTER CONTENT
%----------------------------------------------------------------------------------------

\opening{Monsieur Selem CHARFI,}

Je vous écris à propos de l'offre d'emploi que j'ai reçu de mon encadrant de thèse Florian {\sc Richoux}, dans laquelle est disponible un poste de Docteur en Informatique (section 27). En ce moment j'occupe le poste d'ATER dans le Laboratoire de Sciences du Numérique de l'Université de Nantes, au sein de l'équipe de recherche Théorie, Algorithmes et Applications en Contraintes (TASC). %Mi director de tesis fue Eric Monfroy, director del departamento del informática de la universidad de Nantes.

Je suis passionné de la programmation en général. J'ai toujours été intéressé pour le développement de logiciels applicables, particulièrement pour ceux qui peuvent être implémentés dans des environnements en parallèle, et avec le but de résoudre des problèmes pratiques. Dans le CV ci-joint, vous pouvez trouver que je n'ai pas que de l'expérience dans l'enseignement en différents sujets (y compris des différents langages de programmation, C\#, MATLAB, Javascript, etc..., et Introduction au MPI), mais aussi que j'ai de l'expérience comme développeur des logiciels, principalement en C\# et dans la création et/ou entretien des sites WEB avec ASP.Net. 

Depuis longtemps j'attendais pour l'opportunité d'appliquer toute ma connaissance et mon expérience dans un poste ou je puisse au même temps continuer mon développement professionnel. Je pense que mon expérience comme développeur dans plusieurs entreprises, et mes années de travaille dans des équipes de recherche en plusieurs universités, peuvent influencer positivement votre entreprise. Les années de travail dans ces milieux, m'ont doté d'un sens de responsabilité, et ont fait de moi une personne travailleur, avec des aptitudes pour programmer, apprendre et générer des solutions efficacement. 

%I have enclosed my CV, incluyendo mis publications mas importantes. 

En espérant pouvoir vous exprimer ma motivation dans le cadre d'un entretien, je vous prie de croire, Monsieur, à l'expression de mes salutations distinguées.

\closing{Bien cordialement,}

%----------------------------------------------------------------------------------------

\end{letter}
\end{document}
